\documentclass{article}
\renewcommand{\normalsize}{\fontsize{12}{13}\selectfont} 
\usepackage[landscape]{geometry}
\usepackage{url}
\usepackage{multicol}
\usepackage{amsfonts}
\usepackage{tikz}
\usepackage{amsmath}
\usepackage{listings}
\usepackage{colortbl}
\usepackage{xcolor}
\usepackage{mathtools}
\usepackage{amssymb}
\usepackage{enumitem}
\usepackage[brazilian]{babel}
\usepackage[utf8]{inputenc}
\usepackage{adjustbox}
\usepackage{tcolorbox}
\usepackage{fancyhdr}
\usetikzlibrary {shapes.callouts}
\advance\textheight4in
\advance\textwidth3in
\advance\oddsidemargin-1.5in
\advance\evensidemargin-1.5in
\parindent0pt
\parskip2pt
\newcommand{\hr}{\centerline{\rule{3.5in}{1pt}}}

% Setup the size of paper % 
\geometry{
  paperwidth=850pt,
  paperheight=1100pt,
  margin=25pt
}

\lstset{
  basicstyle=\ttfamily,
  columns=fullflexible,
  showstringspaces=false
}

% Setup the font with style of R %
\lstdefinestyle{Rstyle}{
    language=R,
    basicstyle=\ttfamily\fontsize{13}{13}\selectfont,
    keywordstyle=\color{blue},
    stringstyle=\color{red},
    breaklines=true,
    morekeywords={function,st,github,qgisprocess,sf,fun,run,qgis,tmp,raster,base,algorithms,p,argument,specs,providers,algorithm}
}
\definecolor{mycolor}{HTML}{589632}

% Setup the style of paper %
\fancypagestyle{footerstyle}{
  \fancyfoot[C]{%
    \begin{tikzpicture}[remember picture,overlay]
      \node[fill=mycolor, text=white, minimum width=\paperwidth, minimum height=0.7cm, anchor=south] at (current page.south) {};
      \node[anchor=east, text=white, yshift=0.35cm, xshift=-14.5cm] at (current page.south) {\textcolor{white}{\textbf{Created by Antony Barja}}};
      \node[anchor=center, text=white, yshift=0.35cm, xshift= 14.5cm] at (current page.south) {\textcolor{white}{\textbf{Revised by Floris Vanderhaeghe and updated: 23/06/2023}}};
    \end{tikzpicture}
  }
  \renewcommand{\footrulewidth}{0pt}
  \setlength{\headsep}{0.5cm} 
  }

% ----------------------------------- Cheatsheet ---------------------------------------------- %
\begin{document}
\pagestyle{footerstyle}
\begin{flushleft}
  \begin{adjustbox}{width=850pt}
    \fontsize{40}{30}\selectfont\textbf{QGIS in R with qgisprocess :: CHEAT SHEET}
  \end{adjustbox}
\end{flushleft}

\begin{tikzpicture}[remember picture, overlay]
  \node[anchor=north east, yshift=-12pt, xshift=-12pt] at (current page.north east) {\includegraphics[width=0.12\textwidth]{logo/qgisprocess.png}};
\end{tikzpicture}

\begin{multicols*}{3}
\setlength{\columnsep}{400pt}
\tikzset{
  mybox/.style={
    draw=gray,
    fill=white,
    line width=0.005pt,
    rectangle,
    rounded corners,
    inner sep=10pt,
    inner ysep=5pt
  },
  fancytitle/.style={
    fill=mycolor,
    text=white,
    font=\bfseries
  }
}

% Mission %
\begin{minipage}{\linewidth}
  \vspace{-5pt}
   \textbf{\fontsize{44}{44}\selectfont \textbf{Mission}}\vspace{8pt}\\ 
  The main objective is to provide the R interface for the most popular open-source desktop GIS program like QGIS. This package is a re-implementation of the functionality provided by the archived \textbf{RQGIS} package, which was partially revived in the \textbf{RQGIS3} package.
\end{minipage}

% Features %
\begin{minipage}{\linewidth}
  \vspace{12pt}
   \textbf{\fontsize{20}{5}\selectfont \textbf{Features}}\vspace{8pt}\\ 
  This package makes it easier to use native functions from QGIS and some from GDAL, GRASS and many others (like SAGA).

\begin{center}
\setlength{\arrayrulewidth}{0.001pt} 
\arrayrulecolor{gray}
\renewcommand{\arraystretch}{1.25} 
\begin{tabular}{|c|c|}
  \hline
  \rowcolor{mycolor} 
  \textcolor{white}{\textbf{Providers}} & \textcolor{white}{\textbf{Algorithms}} \\
  \hline
  qgis &  50 + 242 ( c ++) + 1 (3D) \\
  \hline
  gdal & 56 \\
  \hline
  grass & 304 \\
  \hline
  third-party providers & x \\
  \hline
  Total counts & 653 + x \\
  \hline
\end{tabular}
\end{center}  
\end{minipage}

\begin{lstlisting}[style=Rstyle]
 # Show a tibble with processing providers
 > qgis_providers( ) 
 # Show a tibble with algorithms
 > qgis_algorithms( ) 
 # Search algorithms using regular expressions
 > qgis_search_algorithms(
        algorithm = <x>,
        provider = <y>,
        group = <z> 
        )    
\end{lstlisting}

% Installation %
\begin{minipage}{\linewidth}
\vspace{10pt}
   \textbf{\fontsize{20}{5}\selectfont \textbf{Installation}\vspace{5pt}} 
    \begin{lstlisting}[style=Rstyle]
 > install.packages('remotes')
 > install_github('r-spatial/qgisprocess')
 > library(qgisprocess)
    \end{lstlisting}
\end{minipage}


\begin{tikzpicture}
  \node[mybox,text width=11cm] (box) 
  {\\
  If needed, specify path to QGIS installation before
  loading qgisprocess:
\begin{lstlisting}[breaklines=true,style=Rstyle]
> options("qgisprocess.path" = "C:/Program Files/QGIS 3.30/bin/qgis_process-qgis.bat")
\end{lstlisting}
  };
  \node[fancytitle, right=0pt,yshift=6pt] at (box.north west) {GNU/Linux, macOS, Windows};
\end{tikzpicture}

\begin{tikzpicture}
  \node[mybox,text width=11cm] (box) 
  {\\
  1.Get started with the installation of docker in your machine.
  
  2.Download the image of geocomputation  
\begin{lstlisting}[breaklines=true,style=Rstyle]
> docker pull geocompr/geocompr:qgis-ext
\end{lstlisting}

 3. Run to image of geocomputation with docker
\begin{lstlisting}[breaklines=true,style=Rstyle]
> docker run -d -p 8786:8787 -v $(pwd):/home/rstudio/data -e PASSWORD=pw geocompr/geocompr:qgis-ext
\end{lstlisting}
 
  };
  \node[fancytitle, right=0pt,yshift=6pt] at (box.north west) {Using docker};
\end{tikzpicture}


% columa 2 %
% Input %
\begin{minipage}{\linewidth}
  \vspace{-5pt}
   \textbf{\fontsize{44}{44}\selectfont \textbf{Input functions}}\vspace{8pt}\\ 
   The package offers new functionalities of Input to have a workflow of an easy manner inside of R.
\end{minipage}
\begin{lstlisting}[style=Rstyle]
 # Show a description of the function to use
 > qgis_show_help(algorithm ='native:creategrid')

 # Show all the parameters of the function
 > qgis_get_argument_specs(algorithm = 'native:creategrid')
\end{lstlisting}
\begin{lstlisting}[style=Rstyle]
 # Run the algorithms
 > qgis_run_algorithm( 
     algorithm = 'native:creategrid',
     TYPE = 4,
     EXTENT = c('794599, 798208, 8931775,8935384'),
     HSPACING = 1000 ,
     VSPACING = 1000,
     CRS = 'EPSG:32717',
     OUTPUT = 'grid'
       )

 # Create a function based on the algorithm to use 
 > grid_fun <- qgis_function('native:creategrid')
 > grid_fun(
    TYPE = 4,
    EXTENT = c('794599,798208,8931775,8935384'),
    HSPACING = 1000,
    VSPACING = 1000,
    CRS = 'EPSG:32717',
    OUTPUT = 'grid'
     )
\end{lstlisting}

% Output functions %
\begin{minipage}{\linewidth}
  \vspace{12pt}
   \textbf{\fontsize{20}{5}\selectfont \textbf{Output functions}}\vspace{8pt}\\ 
qgisprocess give us new functionalities of output for vector, raster and other format file, and it is possible loads it  to our environment work.
\end{minipage}

\begin{lstlisting}[style=Rstyle]
 > qgis_extract_output(result_run_alg, 'OUTPUT')
\end{lstlisting}
\begin{lstlisting}[style=Rstyle]
# A character vector indicating the location of a  temporary file.
 > qgis_tmp_base( )
 > qgis_tmp_file( ".csv" )
 > qgis_tmp_vector( )
 > qgis_tmp_raster( )
\end{lstlisting}

\vspace{145pt}

% Columna 3 %
% Pipe %
\begin{minipage}{\linewidth}
\vspace{-5pt}
\textbf{\fontsize{44}{44}\selectfont \textbf{Pipe integration}}\vspace{-13pt}
\vspace{13pt}
\begin{lstlisting}[style=Rstyle]
 > qgisprocess also provides
 > qgis_run_algorithm_p() 
\end{lstlisting}
that works better in pipelines.
\end{minipage}

\begin{lstlisting}[style=Rstyle]
 > library(sf)
 > system.file(
    'longlake/longlake_depth.gpkg',
    package = 'qgisprocess'
    ) |>
  qgis_run_algorithm_p(
    algorithm = 'native:buffer',
    DISTANCE = 100
    ) |> 
    st_as_sf( ) |>
    plot( ) 

qgis_fun(...)
\end{lstlisting}

% Workflow %
\begin{minipage}{\linewidth}
  \vspace{12pt}
   \textbf{\fontsize{20}{5}\selectfont \textbf{Workflow}}\vspace{8pt}\\ 
  This package makes it easier to use native functions from QGIS and some from GDAL, GRASS and many others (like SAGA).
\end{minipage}

\begin{tikzpicture}
  \node[mybox,text width=11.5cm] (box) 
  {\\
\begin{lstlisting}[breaklines=true,style=Rstyle]
> library(sf)
> grid_fun <- qgis_function('native:creategrid')
  grid_fun(
    TYPE = 4,
    EXTENT = c('409967, 411658, 5083354, 5084777'),
    HSPACING = 400,
    VSPACING = 400,
    CRS = 'EPSG:26920',
    OUTPUT = 'grid'
    ) |>
   st_as_sf() |> 
   select(id) |>
   plot()
\end{lstlisting}
  };
\node[fancytitle, right=0pt, yshift=6pt] at (box.north west) {Vector data};
\end{tikzpicture}
\begin{tikzpicture}
\node[mybox,text width=11.5cm] (box) 
{
\begin{lstlisting}[breaklines=true,style=Rstyle]
 > library(stars)
 > dem <- read_stars(
    system.file(
      'raster/nz_elev.tif',
       package = 'spDataLarge')
      )
 > qgis_run_algorithm(
    algorithm ='sagang:sagawetnessindex',
    DEM = dem,
    TPI = 'tpi.sdat') |> 
    qgis_extract_output('TWI') |> 
    st_as_stars() |> 
    plot(col = cptcity::cpt(pal = 'ocal_blues'))
\end{lstlisting}
};
\node[fancytitle, right=0pt,yshift=6pt] at (box.north west) {Raster data};
\end{tikzpicture}
\end{multicols*}
\end{document}
